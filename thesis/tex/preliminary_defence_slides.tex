\documentclass[12pt, xcolor={dvipsnames}]{beamer}
\mode<presentation>{
  \usetheme{Madrid}
  % or ...
  \usecolortheme[named=OliveGreen]{structure}
  \setbeamercovered{transparent}
  % or whatever (possibly just delete it)
}
\usepackage[T2A,T1]{fontenc}
\usepackage[utf8x]{inputenc}

\usepackage[russian]{babel}

\usepackage{graphics}

\usepackage{wrapfig}
\usepackage{tikz}

\setlength{\parskip}{\baselineskip} 
%\usepackage[T1]{fontenc}
% or whatever

%\usepackage[latin1]{inputenc}
% or whatever

%\usepackage{times}
%\usepackage[T1]{fontenc}
% Or whatever. Note that the encoding and the font should match. If T1
% does not look nice, try deleting the line with the fontenc.


\title[] % (optional, use only with long paper titles)
{Анализ поверхности взаимодействия белков и поиск наиболее значительных позиций методом in silico Ala-scan}

%subtitle
%{Include Only If Paper Has a Subtitle}

\author[] % (optional, use only with lots of authors)
{
  \texorpdfstring{
	\begin{columns}
	\column{.55\linewidth}
		Магистрант:\\
		Научный руководитель:
	\column{.45\linewidth}
		Татьяна Малыгина, СПбАУ\\
		Павел Яковлев, BIOCAD
	\end{columns}
%	\\[30pt]
%    \begin{columns}
%        \column{.55\linewidth}
%        Место прохождения практики:
%        \column{.45\linewidth}
%        BIOCAD
%    \end{columns}
   }
   {\& }
}

% - Give the names in the same order as the appear in the paper.
% - Use the \inst{?} command only if the authors have different
%   affiliation.

%\institute[Universities of Somewhere and %Elsewhere] % (optional, but mostly needed)
%{
%  \inst{1}%
%  кафедра МиИТ, СПбАУ
%}
% - Use the \inst command only if there are several affiliations.
% - Keep it simple, no one is interested in your street address.

\date[DIPLOMA 2015] % (optional, should be abbreviation of conference name)
{СПбАУ, 2015}
% - Either use conference name or its abbreviation.
% - Not really informative to the audience, more for people (including
%   yourself) who are reading the slides online



% If you wish to uncover everything in a step-wise fashion, uncomment
% the following command: 

%\beamerdefaultoverlayspecification{<+->}
\setbeamertemplate{footline}[frame number]

\begin{document}
\begin{frame}
  \titlepage
\end{frame}
%\begin{frame}{Outline}
%  \tableofcontents
  % You might wish to add the option [pausesections]
%\end{frame}
\section{Введение}
\begin{frame}{Белки и энергия}
1-2
\end{frame}

\begin{frame}{Энергетически горячие аминокислотные остатки}
3
\end{frame}
\begin{frame}{Аланиновое сканирование}
основная идея
\end{frame}
\begin{frame}{Ala-scan in silico}
в чем идея, мб розетта аласкан протокол
\end{frame}
\begin{frame}{Почему проводят не по всем позициям}
как выбирают

методы
\end{frame}

\begin{frame}{Минусы первого метода}
1. картинка с примером
\end{frame}
\begin{frame}{Гомология}
минусы второго метода
разнородные данные в базах, собираются вручную по результатам химического аласкана

частичное несоответствие структур в базах

разные базы. есть даже вики!
\end{frame}
\begin{frame}
вывод - не существует универсального способа поиска 

Формализуем задачу:
что берем для оценки сцепленности - энергию
\end{frame}

\section{Поиск регионов}
\begin{frame}{Триангуляция Делоне и двойственные графы}
\end{frame}
\begin{frame}{Алгоритм поиска протяженных регионов,}{ потенциально содержащих ,,энергетически горячие точки''}
Включим в состав множества протяженных регионов, содержащих ,,энергетически горячие аминокислотные остатки'', следующее:
\begin{itemize}
\item аминокислоты, образующие ,,интерфейс'' взаимодействия с парной цепочкой или белком (с использованием отсечки по расстоянию от второй цепочки)
\item аминокислоты, образующие поверхность ,,карманов'', находящихся в области взаимодействия пары белков
\item не-гидрофобные аминокислоты, являющиеся соседними по отношению к аминокислотам, образующим интерфейс
\item если интерфейс взаимодействия образован петлями, то добавим все аминокислоты, образующие петли 
\end{itemize}


\end{frame}
\begin{frame}{,,Интерфейс''}
\begin{itemize}
\item определяем множество треугольников выпуклой оболочки, для которых хотя бы одна вершина удалена от центров атомов второй цепочки не больше, чем на выбранное значение отсечки
\item Далее расширяем интерфейс
\begin{itemize}
\item шаг 1: добавляем к интерфейсу все треугольники выпуклой оболочки, содержащие атомы аминокислот, которые уже туда попали
\item шаг 2: продлеваем регион до границы гидрофобности
\item шаг 3: продлеваем регион за границы гидрофобности на 1 аминокислоту.
\end{itemize}
\end{itemize}

В результате у нас есть одна или нескольких протяженных связных областей выпуклой оболочки, по которым можно восстановить аминокислоты.
\end{frame}

\begin{frame}{Обработка ,,карманов''}
картинка как в caver

схема графа

поиск в ширину

мб псевдокод?

\end{frame}
\begin{frame}{Петли}
Известно, что петли являются наиболее подвижным и гибким структурным элементом вторичной структуры белка. Если область белок-белкового взаимодействия содержит петлю, вероятно, что энергетически горячий аминокислотный остаток, изменение которого приведет к существенному изменению всей петли в пространстве и, следовательно, существенному изменению взаимодействия.

Из этих соображений имеет смысл включать в область поиска все аминокислоты, образующие петли, частично выходящие в область белок-белкового взаимодействия.

КАРТИНКА?

Перед добавлением петель треугольники триангуляции преобразуются в фрагменты последовательности аминокислот.

\end{frame}
\section{Аланиновое сканирование}

\begin{frame}{Ala-scan in silico с фильтрацией данных}
\textbf{Полученный алгоритм аланинового сканирования} основан на Rosetta alascan protocol и образован следующей последовательностью действий:
\begin{itemize}
\item читаем 2 цепочки атомов из PDB,
\item выбираем аминокислоты одной из цепочек с помощью приведенного выше алгоритма поиска,
\item для этих аминокислот пробуем провести мутагенез с использованием метода Монте-Карло выводим те, изменение которых привело к существенным изменениям свободной энергии системы.
\end{itemize}
\end{frame}

\section{Результаты}
%здесь должны быть скриншоты из PyMol,
%на которых видно выделение областей в отдельный mesh

\begin{frame}{}
Вопросы?
\end{frame}

\end{document}

