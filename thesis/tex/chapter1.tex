\chapter{Постановка задачи и обзор существующих методов}

\section{Аланиновое сканирование}
Аланиновое сканирование (ала-скан) - метод для определения аминокислот в составе белка, играющих важную роль в сохранении его функций, стабильности или формы.
\subsection{Аланиновое сканирование in vitro/in vivo}
Проблемы ala-scan in vitro/in vivo:
\begin{itemize}
\item Большое пространство поиска.
\item Сложность синтеза библиотек: необходим индивидуальный подход!
\item Высокая стоимость.
\end{itemize}
\subsection{Аланиновое сканирование in silico: решаемые задачи и границы применимости}
\section{Стандартные методы выбора аминокислот для аланинового сканирования in silico}

Как правило, при компьютерном моделировании аланинового сканирования применяют предварительную фильтрацию аминокислот. Используются два основных метода:
\begin{itemize}
\item отсечка по расстоянию ~\cite{kortemme2004}: мутации подвергаются только те аминокислоты, которые расположены вблизи области взаимодействия пары белков;

\item выбор по гомологии: аминокислоты выбираются исходя из имеющихся экспериментальных результатов аланинового сканирования, проводившихся на близких (в эволюционном смысле) последовательностях.
\end{itemize}

Посмотрим, насколько эффективны эти подходы.

\subsection{Использование отсечки по расстоянию}
Эксперимент
\begin{itemize}
\item Рассмотрим базу данных с информацией о результатах эспериментов по аланиновому сканированию ASEdb.
\item Найдем объекты со ссылкой на Protein Data Bank.
\item Среди всех таких объектов, найдем те, в которых есть аминокислоты, мутация которых приводит к существенному изменению свободной энергии комплекса ($\geq 1$ ккал/моль)
\item Посмотрим, всегда ли они удалены от интерфейса в пределах стандартно используемой отсечки (в качестве примера возьмем расстояние, не превышающее 8 \AA{}).
\end{itemize}

