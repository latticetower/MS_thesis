\graphicspath{{../images/algorithm/}}
\chapter{Экспериментальные результаты}
%\todo{глава не готова}
\section{Пример работы итеративного алгоритма выбора протяженных регионов}
Для оценки корректности работы алгоритма на каждом из этапов проводился визуальный контроль выбираемых аминокислот с использованием средств визуализации приложения PyMol ~\cite{pymol}.

Далее приведены несколько скриншотов с L и H цепями антитела (2OSL.pdb, но без молекул воды) с разных ракурсов, на которых выделены разными цветами аминокислоты, добавляемые к региону для аланинового сканирования на разных этапах:
\begin{itemize}

\item желтым цветом изображена L-цепь, на ней голубым цветом выделены аминокислоты, содержащие атомы, которые попали в начальное отобранное множество треугольников выпуклой оболочки;

\item синим цветом выделены аминокислоты, атомы которых попали в выделенную область на этапе поиска карманов;

\item красным цветом выделены участки цепочки, которые были добавлены в результате добавления петель, частично попавших в выделенный регион на одном из предыдущих этапов.
\end{itemize}


\resizebox{0.8\textwidth}{!}{
\includegraphics{loops1.png}
}

\resizebox{0.8\textwidth}{!}{
\includegraphics{loops2.png}
}

На последнем из приведенных скриншотов в нижней части изображения видна петля, которая удалена от интерфейса взаимодействия, но при этом один из атомов ее аминокислот по удачному стечению обстоятельств попал в множество отобранных треугольников выпуклой оболочки, за счет чего петля попала целиком.

%\todo{И это наводит на мысль о том, что изначальный выбор аминокислот не так хорош. Возможно, стоит использовать граф Габриэля? }

\resizebox{0.8\textwidth}{!}{
\includegraphics{loops3.png}
}

%Посередине петля полностью желтая - но это из-за способа, которым добавлялись карманы.

\resizebox{0.8\textwidth}{!}{
\includegraphics{loops4.png}
}

%\newpage
\section{Протокол аланинового сканирования}
% http://www.bakerlab.org/system/files/kortemme04B.pdf
% - картинка отсюда с страницы 3
%\todo{-нужна картинка про rosetta alascan protocol }

%В работе \cite{kortemme2002} показано, что (сказать про функцию энергии для компьютерного моделирования аланинового сканирования)

%\todo{-описание того, в какой части предполагается использовать модиф регионы}
%\vspace{10cm}
\section{Сравнение с другими инструментами}
%Уже обработанные данные есть здесь: \cite{kortemme_alascan_datasets}

%Картинка \cite{benchmark_img} но я не уверена, можно ли ее приводить без разрешения

%\todo{- картинка с схемой работы скрипта}

%\todo{- красивая табличка, с процентом аминокислот в цепочке, попавших в регион, а также попавшими и не попавшими хотспотами.}
%Описание таблички (нужно для корректного вывода в скрипте)
% 1. идентификатор файла pdb
% 2. мутируемая цепь
% 3. цепь\ лиганд, взаимодействие с которым оценивается
% 4. длина цепочки, длина региона, число замаскированных амк
% 5. 
%\\

%\todo{-написать скрипт, который для данного pdb и пары цепочек проводит ala-scan по всем позициям и сравнивает методы, которые используются для фильтрации аминокислот.}

%\todo{сделать табличку по данным asedb и bid, и для того, что нет (хотспоты искать с помощью протокола аланинового сканирования, в табличке рисовать циферки). если есть амк, которые не нашлись,}
\vspace{10cm}



