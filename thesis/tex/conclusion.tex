\conclusion
В данной работе был построен алгоритм поиска протяженных регионов, определяющих специфичность белка по отношению к белковому комплексу, основанный на общих представлениях об энергетически значимых аминокислотах, которые оказывают существенный вклад в функцию, стабильность или форму белка, и тем самым формируют такие регионы. 

Было проведено тестирование алгоритма на наборе данных Kortemme \& Baker, показано, что в такие регионы попадают все (или почти все) энергетически значимые аминокислоты, которые были определены экспериментально.

Тем не менее, получаемый таким образом регион, определяющий специфичность -- достаточно обширен; это приводит к низкой  точности такого метода. Низкую точность можно объяснить следующими причинами:
\begin{itemize}
\item задача предсказания энергетически значимых аминокислот является труднорешаемой и трудноформализуемой, на текущий момент нет  инструментов, позволяющих предсказывать энергетически значимые аминокислоты с высокой точностью; 
\item хотя приведенный алгоритм позволяет выявить регионы, определяющие специфичность, задача определения энергетически значимых аминокислот на них  осложняется тем, что энергетическая значимость аминокислоты может быть получена в том числе за счет изменения положения боковых цепей соседних с ней аминокислот, то есть она зависима от окружения.
\end{itemize}

Полученный алгоритм может быть использован  для выбора аминокислот и проведения дальнейшего аланинового сканирования in silico. Он реализован на языке python, с применением библиотек SciPy\cite{scipy} и NumPy\cite{numpy}.

Кроме того, написан плагин для приложения PyMOL~\cite{pymol}, визуализирующий регион, определяющий специфичность, для выбранной структуры.

