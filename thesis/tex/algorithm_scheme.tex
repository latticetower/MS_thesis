\usepackage{tikz}
\usepackage{xparse}

\DeclareDocumentCommand{\algoscheme}{  }{

\begin{tikzpicture}[%
  % common options for blocks:
  block/.style = {draw, fill=green!30, align=center, anchor=west,
              minimum height=0.65cm, inner sep=0},
  % common options for the circles:
  ball/.style = {circle, draw, align=center, anchor=north, inner sep=0},
  line/.style     = { draw, thick, ->, shorten >=2pt },
  ]

% draw the sieves
\node[block,anchor=north,text width=6.4cm] (tray1) at (0,0) {PDB-файл
   {\begin{itemize}
   \item координаты атомов в пространстве и их тип;\\[-5pt]
   \item информация о вторичной структуре.
   \end{itemize}}
   };

\node[block,below right = of tray1,text width=7.4cm] (tray2) {Построение триангуляции Делоне и двойственного графа};
   
\node[block,below = of tray2,text width=7.4cm] (tray3) {Выбор начальных треугольников на границе триангуляции};

\node[block,below = of tray3,text width=7.4cm] (tray4) {Поиск карманов};

\node[block,below = of tray4,text width=7.4cm] (tray5) {Добавление петель и $\beta$-поворотов};

\node[block,below left = of tray5,text width=5.4cm] (tray6) {Позиции для аланинового сканирования};

\node[block, below = of tray6,text width=5.4cm] (tray7) {rosetta alascan protocol};
\node[block, below right= of tray6,text width=4.4cm] (tray8) {PyMOL}; 


\begin{scope} [every path/.style=line]
\path (tray1) -- (tray2); 
\path (tray2) -- (tray3);
\path (tray3) -- (tray4); 
\path (tray4) -- (tray5); 
\path (tray5) -- (tray6); 
\path (tray6) -- (tray7);
\path (tray6) -- (tray8);  
\end{scope}
\end{tikzpicture}
}