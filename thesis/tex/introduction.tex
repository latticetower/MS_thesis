\intro

In silico ala-scan - теоретическая версия экспериментального комбинаторного метода, применяемого органическими химиками для решения простой задачи - понять, какие аминокислоты оказывают существенный вклад в структуру белка. Идея метода состоит в замене одной аминокислоты в цепочке аланином и наблюдении, как такая точечная замена повлияла на свойства белка - по каждой позиции в белке. Если при замене аминокислоты в какой-либо позиции свойство улучшились, то для позиции, в которой это произошло, среди аминокислот ищем ту, при подстановке которой улучшение экстремально.

У ala-scan есть один недостаток: при большой длине белка существенно увеличивается сложность метода, появляется необходимость фильтровать данные и проводить ala-scan не по всем позициям, а выборочно.


В данной работе разработан метод выбора регионов для фильтрации входных данных перед применением теоретического ala-scan для улучшения структуры известного белка (а именно, для достижения лучшей сцепленности между двумя парами цепочек). 
%\cite{yakovlev2014}

%\cite{konf_analiz_belkov}
