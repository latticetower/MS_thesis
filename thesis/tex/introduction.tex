\intro

In silico ala-scan - теоретическая версия экспериментального комбинаторного метода, применяемого органическими химиками для решения простой задачи - понять, какая аминокислота оказывает существенный вклад в структуру белка. Идея метода состоит в замене одной аминокислоты в цепочке аланином и наблюдении, как это повлияло на свойства белка - и далее по всем позициям в белке. Если при замене аминокислоты в какой-либо позиции свойство улучшились, то для позиции, в которой это произошло, среди аминокислот ищем ту, при подстановке которой улучшение экстремально.

У ala-scan есть один недостаток: при большой длине белка существенно увеличивается сложность метода, появляется необходимость фильтровать данные и проводить ala-scan не по всем позициям, а выборочно.


В данной работе разработан метод выбора регионов для фильтрации входных данных перед применением теоретического ala-scan для улучшения структуры известного белка (а именно, для достижения лучшей сцепленности между двумя парами цепочек). 

\todo{переписать то, что дальше - то, что выше почти без изменений начало описания летней практики - мне нравится, как оно написано}


Диссертация состоит из четырех глав. 
во второй главе приведено описание алгоритма, который используется для выбора регионов

в третьей приведено описание используемой модификации протокола аланинового сканирования, причины использования этого протокола, а также сравнение результатов при выборе регионов по-умолчанию или при выборе регионов разработанным методом 

\todo{вспомнить, что я хотела от 4 главы - нужна ли она? и написать нормальное введение}