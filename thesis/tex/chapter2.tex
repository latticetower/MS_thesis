\graphicspath{{../images/algorithm/}}
\chapter{Описание разработанного метода}

\section{Биологическое обоснование}
Как видно из предыдущей главы,  в настоящий момент при выборе аминокислот, подвергаемых точечному аланиновому мутагенезу, не проводится какой-либо сложный анализ. Выбор аминокислот с отсечкой по расстоянию -- не универсален, величину отсечки необходимо подбирать индивидуально для каждого взаимодействующего комплекса. Предсказание аминокислот позволяет ,,угадать'' место точечной мутации только на основании уже имеющихся данных, при их отсутствии или недостатке этот метод не работает, его качество существенно зависит от имеющейся информации о похожих структурах -- их качества и точности, при этом методов с высокой точностью предсказания нет. 

Попробуем предложить метод выбора аминокислот, основанный на известных общих свойствах белков и белковых комплексов. Для этого поймем, какие свойства необходимо учитывать.

% О-кольца
% фрагменты петель
% гидрофобность
% карманы

%\newpage
\subsection{Гидрофобность}
Широко известна и подтверждена гипотеза о том, что фолдинг белков происходит под влиянием свойств аминокислот: гидрофобные аминокислоты стремятся оказаться внутри, поскольку отталкиваются от молекул воды, в растворе которой находится белок~\cite{hydrophobic}.

Похожим образом формируется интерфейс взаимодействия двух цепочек, связь между ними образуется за счет гидрофобных взаимодействий~\cite{hydrophobic2chain}: в область связывания попадают в первую очередь гидрофобные аминокислоты, находящиеся на поверхности цепочки белка.

Логично предположить, что усиление взаимодействия между парой цепочек может произойти при изменении гидрофильной аминокислоты вблизи интерфейса взаимодействия цепочек на гидрофобную, ослабление -- при изменении гидрофобной аминокислоты в области интерфейса на гидрофильную.

%\newpage
\subsection{Карманы}
Исследование энергетически значимых аминокислот в области интерфейса взаимодействия белков приведено в работе~\cite{pockets2004}. В соответствии с ним, комплементарные карманы -- такие карманы, которые при формировании комплекса оказываются заполнены герметично -- чаще содержат энергетически значимые или структурно консервативные аминокислоты по сравнению с карманами, заполняемыми при формировании комплекса не полностью.

И хотя в приведенной работе выводы основаны на небольшом числе наблюдений, они хорошо соотносятся с интуитивно понятным утверждением: ,,энергетическая значимость'' аминокислоты может быть выражена в ее существенном влиянии на форму поверхности белка. Тогда изменение аминокислоты в глубине кармана может привести к изменению его формы, сделав незаполненный, некомплементарный карман -- комплементарным, или наоборот. Поэтому в состав регионов, определяющих специфичность, следует добавить аминокислоты, образующие внутреннюю поверхность карманов.
%\newpage
\subsection{Петли}
Попытка обобщить понятие энергетически горячего аминокислотного остатка до регионов проводится в работе~\cite{loops2014}. В ней вводится понятие ,,горячей петли'' -- такого короткого (до 10 аминокислот) фрагмента  цепочки белка, содержащей более 2 энергетически значимых аминокислотных остатков. В упомянутой статье проводится анализ таких петель по данным ASEdb~\cite{asedb2001} и попытка каким-то образом сгруппировать их и сделать выводы о том, какие аминокислоты чаще встречаются в таких ,,горячих петлях''.

Идея о возможности обобщения основана на понятии о гибкости петель, их способности существенно менять положение в пространстве в зависимости от поворота образующих их аминокислот.

%о которой известно, что она моЧто там происходит: взята ASEdb, рассмотрены короткие фрагменты петель, содержащие энергетически горячие аминокислоты и приведена их какая-то классификация.\newpage
\subsection{Поиск по гомологии}
Уместно учесть информацию об энергетически значимых аминокислотах в тех ситуациях, когда одна из рассматриваемых цепочек комплекса близка (в эволюционном смысле) к цепочке комплекса, для которого результаты аланинового сканирования известны, а парная цепочка или лиганд, с которой (которым) оценивается взаимодействие, совпадает.

Такое расширение алгоритма может помочь учитывать ситуации, когда энергетический вклад аминокислоты меняется (в большую или меньшую сторону) за счет не мутации, а поворота, происходящего под воздействием изменившегося положения аминокислот, принадлежащих удаленному от интерфейса фрагменту цепочки, сдвинувшемуся в результате точечной модификации (визуально это может выглядеть, например, как каскад петель). 



%\newpage
\section{Алгоритм выбора протяженных регионов}
Учтем приведенные наблюдения, построив итеративный алгоритм выбора регионов (см.рисунок \ref{fig:algo_scheme}).
\begin{figure}
\resizebox{\linewidth}{!}{
\algoscheme
}
\caption{\small{Предлагаемый алгоритм выбора регионов, определяющих специфичность }}
\label{fig:algo_scheme}
\end{figure}
В качестве входных данных для алгоритма будем рассматривать множество координат атомов каждой из пары взаимодействующих цепочек с известными радиусами Ван-дер-Ваальса.

Зафиксируем цепочку, для которой хотим выбрать аминокислоты для последующего аланинового сканирования. 

По точкам, соответствующих центрам образующих эту цепочку атомов, построим триангуляцию Делоне. Для нее построим двойственный граф, вершинам которого сопоставим треугольники триангуляции, ребрам - переход между двумя смежными треугольниками, через которые может пройти  шар фиксированного радиуса. Начиная от треугольников выпуклой оболочки триангуляции, вершины которых расположены вблизи точек - центров атомов другой цепочки комплекса, будем проводить поиск тех карманов, выход из которых может осуществляться через выбранные треугольники выпуклой оболочки. Затем  восстановим позиции аминокислот, атомы которых находятся в вершинах выбранных на предыдущих этапах треугольников. Если какая-либо из этих позиций находится на одном из $\beta$-поворотов или на одной из петель, дополним ее другими позициями этого элемента вторичной структуры.   

Полученное таким образом множество позиций аминокислот будем использовать для последующего аланинового сканирования или дальнейшего анализа.


\newpage
\subsection{Входные данные}

%На входе PDB~\cite{pdb} файл.
%(полстраницы, возможно, с картинкой)
%\vspace{10cm}
\begin{figure}
%\resizebox{0.8\textwidth}{!}{
\includegraphics[width=\linewidth]{atom_in_pdb.png}
%}
\caption{\small{На снимке экрана показан внешний вид секции ATOM в текстовом редакторе.
 }}
\label{fig:atom_in_pdb}
\end{figure}


PDB файл ~\cite{pdb} содержит информацию о белке, сгруппированную по следующим признакам: 
\begin{enumerate}
\item пространственное положение атомов аминокислот (секция ATOM):
включает информацию о координатах каждого из атомов, образующих аминокислоты, в свою очередь образующих цепочки белка или белков, вид атома, другие важные характеристики. Координаты упорядочены по положению аминокислот в цепочке.

\item информацию о вторичной структуре цепочек (секции HELIX, SHEET).
\end{enumerate}


%\newpage\section{Выбор протяженных регионов}
%Для того, чтобы выбрать протяженные регионы для аланинового сканирования, построим итеративный алгоритм.

%Включим в состав множества протяженных регионов, содержащих ,,энергетически горячие аминокислотные остатки'', следующее:
%\begin{itemize}
%\item аминокислоты, образующие ,,интерфейс'' взаимодействия с парной цепочкой или белком (с использованием отсечки по расстоянию от второй цепочки)
%\item аминокислоты, образующие поверхность ,,карманов'', находящихся в области взаимодействия пары белков
%\item не-гидрофобные аминокислоты, являющиеся соседними по отношению к аминокислотам, образующим интерфейс
%\item если интерфейс взаимодействия образован петлями, то добавим все аминокислоты, образующие петли 
%\end{itemize}


%\newpage\subsection{Формализация задачи}
%\begin{frame}{Formal definitions - I}
%\begin{itemize}
%\item Protein molecule - a set of spheres S. Each sphere $s \in S,\,s=s(c_s, r_s)$ has center point $c_s$ and radius $r_s$, ,,atom'' corresponds to sphere.
%\item the Euclidean distance $D(x,\,s)$ 
%of a point $x$ from the surface of a sphere $s$:
%$$
%D(x,\,s) = || x - c_s || - r_s
%$$
%\item The minimal distance from the point x to the nearest sphere (atom) is given by the function r (x):
%$$
%r(x) = \min \{D(x,s) | s \in S \}
%$$
%\end{itemize}
%\end{frame}

\subsection{Маскировка аминокислот}
В качестве предварительной обработки добавим возможность маскировки аминокислот -- исключения из рассмотрения тех фрагментов цепочек, аминокислоты на которых нам не интересны или которые изменять не желательно. 

В качестве примера возьмем случай, когда мы хотим усилить сцепленность между легкой и тяжелой цепями антитела, но при этом не хотим никак повлиять на его специфичность -- в этом случае уместно исключить из рассмотрения CDR-регионы. Противоположная ситуация -- добиться лучшей аффинности с известным антигеном, не нарушив при этом структуру FR-регионов антитела.

Маскируемые регионы в приведенных примерах могут быть получены, например, в результате алгоритма аннотирования, приведенного в работе \cite{yakovlev2014}.

%\todo{или идея в том, что мы не рассматриваем CDR потому что они гипервариабельные, и изменить одну амк на них технически нереально?}
%Исходя
\subsection{Триангуляция Делоне}
Рассмотрим объединение сфер $S$, каждая из которых (обозначим ее $s$) характеризуется координатами своего центра $c_s=(x_s, y_s, z_s)$ в трехмерном пространстве и радиусом $r_s$:
$$
s=s(c_s,\,r_s),\quad s\in S
$$
Каждому атому цепочки белка будем сопоставлять такую сферу с координатами центра, совпадающими с координатами атома в PDB-файле, и радиусом, соответствующим радиусу Ван-дер-Ваальса для данного атома, про все атомы цепочки будем говорить, что вместе они образуют такое объединение сфер.

%\begin{frame}{Formal definitions - I}
%\begin{itemize}
%\item Protein molecule - a set of spheres S. Each sphere $s \in S,\,s=s(c_s, r_s)$ has center point $c_s$ and radius $r_s$, ,,atom'' corresponds to sphere.
%\item the Euclidean distance $D(x,\,s)$ 
%of a point $x$ from the surface of a sphere $s$:
%$$
%D(x,\,s) = || x - c_s || - r_s
%$$
%\item The minimal distance from the point x to the nearest sphere (atom) is given by the function r (x):
%$$
%r(x) = \min \{D(x,s) | s \in S \}
%$$
%\end{itemize}
%\end{frame}
%\newpage
\subsection{Построение двойственного графа}
%\begin{itemize}
%\item Рассматриваем одновременно 2 цепочки, образующие белковый комплекс.
%\item Начнем с построения выпуклой оболочки и триангуляции Делоне для каждой из них, будем искать протяженные регионы с энергетически горячими аминокислотными остатками  на одной из них. Строить будем по центрам атомов, формирующих аминокислоты цепочки.
%\item Выберем все треугольники выпуклой оболочки, в которых хотя бы одна вершина удалена от некоторых атомов второй цепочки не далее, чем на выбранное (фиксированное) значение отсечки.
%\end{itemize}


%\begin{itemize}
%\item Берется структура в PDB.
%\item Центры атомов рассматриваются как точки. По ним строится триангуляция Делоне (с помощью scipy.delaunay - обертки над алгоритмом qhull).
%\item Берется невзвешенная триангуляция -- по тем же соображениям, по которым она используется в CAVER (эвристика)\cite{caver2007}.
%\end{itemize}
%Я сопоставляю графу триангуляцию по тому же принципу, как в статье по CAVER:


%\begin{figure}[h]
%\resizebox{!}{0.3\textheight}{
%\includegraphics{image4_caver.png}
%}

%\caption{\small{[Computation of tunnels in protein molecules using
%Delaunay triangulation, P.Medek, et al., 2007]
% }}
%\label{fig:image4_caver}
%\end{figure}

%Треугольники триангуляции -- вершины графа, ребра графа -- общие стороны треугольников (с ограничением снизу на длину).

%Дополнительно используется такой же граф по треугольникам выпуклой оболочки (без ограничения снизу на длину стороны, просто по смежным по стороне треугольникам).

%Выбирается начальное множество треугольников выпуклой оболочки. Сейчас берутся треугольники, вершины которых недалеко (в смысле отсечки по расстоянию) от второго белка.

%\newpage
\subsection{Псевдо-интерфейс}
%Включим в состав множества протяженных регионов, содержащих ,,энергетически горячие аминокислотные остатки'', следующее: аминокислоты, образующие ,,интерфейс'' взаимодействия с парной цепочкой или белком (с использованием отсечки по расстоянию от второй цепочки)
%\todo{вспомнить, что я хотела сказать таким заголовком}
%\begin{itemize}
%\item определяем множество треугольников выпуклой оболочки, для которых хотя бы одна вершина удалена от центров атомов второй цепочки не больше, чем на выбранное значение отсечки
%\item Далее расширяем интерфейс
%\begin{itemize}
%\item шаг 1: добавляем к интерфейсу все треугольники выпуклой оболочки, содержащие атомы аминокислот, которые уже туда попали
%\item шаг 2: продлеваем регион до границы гидрофобности
%\item шаг 3: продлеваем регион за границы гидрофобности на 1 аминокислоту.
%\end{itemize}
%\end{itemize}


%В результате у нас есть одна или нескольких протяженных связных областей выпуклой оболочки, по которым можно восстановить аминокислоты.
\newpage
\subsection{Поиск карманов}
%\todo{переписать}
Поиск карманов, каналов и полостей в макромолекулах является хорошо изученной задачей \cite{alpha_shapes1995, alpha_shapes1998, caver2007, ppi_kim2006}.  

%\begin{itemize}
%\item Начинаем с множества отобранных треугольников выпуклой оболочки (черным цветом)
%\item Ищем в ширину с ограничениями
%\item Ходим только по тем треугольникам, для которых ближайший треугольник выпуклой оболочки -- один из отобранных треугольников выпуклой оболочки, а не какой-либо из других треугольников выпуклой оболочки.
%\item Ближайший треугольник == в смысле расстояния до ближайшей из вершин он ближе, чем другие треугольники выпуклой оболочки
%\end{itemize}
%\cite{alpha_shapes1995, alpha_shapes1998}
%\newpage
\subsection{Расширение интерфейса. Добавление петель}

Перед добавлением петель треугольники триангуляции преобразуются в фрагменты последовательности аминокислот, продлеваем их, используя информацию о вторичной структуре.



Сейчас вторичная структура определяются на основе вывода DSSP \footnote{используется обертка к DSSP в составе biopython}.

Берутся непрерывные фрагменты структуры, которые не определяются как $\alpha$-спирали или $\beta$-листы, выбираю среди них те, в которые попадает какая-либо аминокислота, добавляются к выделяемому контуру.

%\todo{\item Я беру непрерывные фрагменты структуры, которые не определяются как альфа-спирали или бета-листы, выбираю среди них те, в которые попадает какая-либо аминокислота, ну и добавляю их к выделяемому контуру - технической сложности никакой.}




%\begin{itemize}


%\todo{\item вариантов 2: либо не добавлять петли целиком, вместо этого добавлять только такие фрагменты (в статье они приведены), либо как-то их отмечать на выделенных петлях + отмечать на удаленных петлях, которые в выделенный регион не попали.}

% \todo{\item про HMM думаю, напишу через пару дней.}

% \todo{\item вообще если получится, я разберусь как считать взвешенную триангуляцию Делоне и с помощью нее искать карманы.}


%\end{itemize}
\section{Код}
