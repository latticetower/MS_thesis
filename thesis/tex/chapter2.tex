\chapter{Описание разработанного метода}

\section{Общий обзор}
В данном разделе мы опишем общую идею разработанного метода.

\subsection{Общие соображения}
Включим в состав множества протяженных регионов, содержащих ,,энергетически горячие аминокислотные остатки'', следующее:
\begin{itemize}
\item аминокислоты, образующие ,,интерфейс'' взаимодействия с парной цепочкой или белком (с использованием отсечки по расстоянию от второй цепочки)
\item аминокислоты, образующие поверхность ,,карманов'', находящихся в области взаимодействия пары белков
\item не-гидрофобные аминокислоты, являющиеся соседними по отношению к аминокислотам, образующим интерфейс
\item если интерфейс взаимодействия образован петлями, то добавим все аминокислоты, образующие петли 
\end{itemize}

\section{Детальное описание}
В данном разделе мы дадим исчерпывающее описание алгоритма вместе с необходимыми для реализации деталями.
\subsection{Построение графа}
\begin{itemize}
\item Рассматриваем одновременно 2 цепочки, образующие белковый комплекс.
\item Начнем с построения выпуклой оболочки и триангуляции Делоне для каждой из них, будем искать протяженные регионы с энергетически горячими аминокислотными остатками  на одной из них. Строить будем по центрам атомов, формирующих аминокислоты цепочки.
\item Выберем все треугольники выпуклой оболочки, в которых хотя бы одна вершина удалена от некоторых атомов второй цепочки не далее, чем на выбранное (фиксированное) значение отсечки.
\end{itemize}

\subsection{Псевдо-интерфейс}
\begin{itemize}
\item определяем множество треугольников выпуклой оболочки, для которых хотя бы одна вершина удалена от центров атомов второй цепочки не больше, чем на выбранное значение отсечки
\item Далее расширяем интерфейс
\begin{itemize}
\item шаг 1: добавляем к интерфейсу все треугольники выпуклой оболочки, содержащие атомы аминокислот, которые уже туда попали
\item шаг 2: продлеваем регион до границы гидрофобности
\item шаг 3: продлеваем регион за границы гидрофобности на 1 аминокислоту.
\end{itemize}
\end{itemize}


В результате у нас есть одна или нескольких протяженных связных областей выпуклой оболочки, по которым можно восстановить аминокислоты.

\subsection{Расширение интерфейса. Добавление петель}
Перед добавлением петель треугольники триангуляции преобразуются в фрагменты последовательности аминокислот, продлеваем их, используя информацию о вторичной структуре.

